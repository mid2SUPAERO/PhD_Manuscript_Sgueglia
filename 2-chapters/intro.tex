\chapter*{Introduction}
\markboth{Introduction}{Introduction}
\addcontentsline{toc}{chapter}{Introduction}  
\mtcaddchapter 

Since its birth in '30s, air transport sector has been continuously growing each year. 
In last decade, thanks to the entry of new commercial area, such as south America or western Asia, the trend has become exponential. 
According to perspective from Boeing and Airbus, the two main leaders in the sector, the number of aircraft flying will be increased of about 39000+ units in 20.

This growth poses a key problem when looking at the emission level.
According to 2015 data, aviation accounts for the 3\% of total CO\textsubscript{x} and NO\textsubscript{x} emissions.
Without any action taken, predictions indicate that this percentage will reach values between 30 and 50\% in a 20 years horizon. 
To avoid such scenario, a sustainable growth must be found. 

Several consortium and agency, spreaded between America and Europe, have published their objectives for emission reduction in near and long term, up to 2050 horizon. 
All these perspectives are very aggressive, and it is commonly recognised that they cannot been achieved through the incremental approach that has been used so far in aviation. 
The classical ``tube-and-wing'' (TAW) configuration has been indeed developed for more than 70 years, and it offers only small potential gains: it is then necessary to introduce a new paradigm in aircraft design, a disruptive concept, based on innovative technologies, to drastically reduce emission. 
The study of a disruptive concept takes place at conceptual design level, where different configurations are defined and sized in order to downselect the most promising ones for further developments. 
Thus thus Ph.D. proposes to develop a Multidisciplinary Design and Optimisation (MDAO) process tailored to the design of an unconventional configuration that would match stringent environmental goals. 

In literature, two main areas of research have been considered: technologies at airframe level, focusing on more aerodynamically efficient architectures, and technologies at propulsive level, deploying hybrid and electric propulsive chain. 
Given the objectives, a most likely viable aircraft must benefit from a combination of both technologies, leading to an optilised overall configuration including hybrid powerplant. 
The main problem when dealing with such innovative concepts concerns the models: the classical methods proposed by aircraft design handbooks are well suited for a conventional configuration, but lose their validity when the concept is drastically different, as in the case of hybrid propulsion or integrated airframe concepts. 
From this consideration the main problem to be solved can be stated as following: \textbf{``How can unconventional configurations be investigated at conceptual design level''?}
This dissertation proposes an answer to this question in 4 steps.

Chapter~\ref{chap1:state_art} reviews the state of the art, focusing on the works of major interest for this research. 
After a brief introduction to better frame the environmental context, the conceptual aircraft design procedure is presented.
Then, innovative key technologies are explored: one of the most promising technology found in literature is the electric and hybrid propulsion. 
This technology, which has been applied at automotive industry and on some general aviation aircraft, enables a total or partial mission electrification, greening the air and the reduction of emission. 
Also, such propulsion concepts enable the increase of propulsive and aerodynamic efficiency, through distribution of thrust and/or boundary layer ingestion.. 
Both these aspects will be detailed. 
At airframe level, the Blended Wing-Body (BWB) is identified as a promising concept with high aerodynamic efficiency. 
At the end of the review, a promising integrated concept is identified in the Blended Wing-Body (BWB) with distributed electric propulsion.
Following the statement of the research problem, it is found that aircraft design must go towards a Multidisciplinary Design Optimisation approach, since it provides the necessary tools to establish a tradeoff for unconventional configuration, without neglecting any of the possible key interactions. 

Chapter~\ref{chap2:fast_base_mdo} presents the development of an aircraft optimisation sizing platform, based on the integration of the code FAST within OpenMDAO. 
FAST, which stands for Fixed-wing Aircraft Sizing Tool, is a code developed at ONERA and ISAE-Supaero for the preliminary design of large passenger aircraft. 
OpenMDAO, instead, is a MDAO platform, developed at NASA Glenn Research Center, which has been used for a large variety of MDO problems in aeronautics.
After a presentation of both tools, with features and drawbacks, the integration of FAST and OpenMDAO is desribed.
A suitable MDO architecture for aircraft design problem is identified, and the processus carried out in the integration is detailed.
The developed code is used for the optimisation of the Airbus A320 test case, so that gains of the optimisation process would be highlighted.
Subsequently, the code has been used to define a set of reference aircraft, considering entry into service 2035. 
Results will be used later to evaluate performance of the proposed disruptive concept. 

The BWB with distributed electric propulsion concept is based on the integration of two different innovative technologies: a new hybrid powerplant at propulsive level and a non conventional architecture at airframe level. 
To better quantify the impact of each of these aspects on the overall design, they are individually treated considering a conventional tube-and-wing aircraft featuring distributed electric propulsion and a BWB mouting conventional air-breathing engines. 
The coupling between the two finally leads to the proposed concept of BWB with distributed electric propulsion. 
The following two chapters reflects this approach.

Chapter~\ref{chap3:hybrid_dep_exploration} is dedicated to the methodology and the exploration of the first concept, tube-and-wing aircraft with distributed electric propulsion. 
The first part of the chapter focuses on the the methodology: the overall propulsive chain is defined at system level, and then models related to each electric component are presented. 
Afterwards, these models are integrated within the conceptual design loop and required modifications to the disciplines are detailed.  
Finally, the revised sizing process is included in the MDO formulation defined in Chapter~\ref{chap2:fast_base_mdo}, to finally converge towards an optimisation sizing framework for an hybrid aircraft. 
The second part of the chapter presents the sizing results for this concept: at first only the sizing loop is applied, to assess its quality in dealing with the innovative powerplant.
Afterwards, optimisation results are presented, varying also the number of engines, in order to establish a possible tradeoff. 
Aircraft performances are evaluated against the reference aircraft, defined in Chapter~\ref{chap2:fast_base_mdo}. 
A Pareto frontier is also generated, comparing two different optimisation algorithms, one gradient-based and another one gradient-free, in order to assess the reduction in computational cost due to the utilisation of derivatives in the optimisation process. 

Chapter~\ref{chap4:bwb_exploration} presents the methods that have been implemented for the Blended Wing-Body sizing. 
First, the research strategy is detailed: in order to tackle the lack of reference data for the BWB architecture, a multi-fidelity analysis is carried out at disciplinary level, so that fast but reliable methods can be identified and used in the conceptual design stage. 
Once more, revised models are integrated within the sizing loop, and the resulting procedure is included in the MDO formulation. 
Finally have a MDAO framework tailored to BWB with DEP is obtained. 
The first part of the results relates only to application of the sizing loop, considering conventional engines, in order to assess the benefits coming solely from the BWB architecture. 
Afterwards, the powerplant modeled in Chapter~\ref{chap3:hybrid_dep_exploration} is combined with this architecture, and the BWB with distributed electric propulsion is explored. 
In this case too, performances are evaluated against the reference aircraft EIS 2035. 
A Pareto frontier is also obtained, to establish tradeoff for the proposed disruptive concept. 

This dissertation is concluded with an overall revision of the work and possible further development to be made on the Multidisciplinary Design Analysis and Optimisation process. 