\begin{vcenterpage}

\addcontentsline{toc}{chapter}{Summary}
\mtcaddchapter  

\noindent\rule[2pt]{\textwidth}{0.5pt}

{\large\textbf{Abstract ---}}
The increase of air traffic in the last decades and its projections pose a key challenge towards the carbon neutral growth objective. 
To cope with this societal goal, there is a need for disruptive air transport aircraft concepts featuring new technologies with low environmental impact. Such future air vehicle relies on the various interactions between systems, disciplines and components. 
This Ph.D. research thus focuses on the development of a methodology dedicated to the exploration and performance evaluation of unconventional configurations using innovative propulsion concepts. 
The use case to be considered is the optimization at conceptual level of a Blended Wing-Body with distributed electric propulsion, a promising concept which combines high aerodynamic performances and benefits from electric propulsion.

The optimization process based on FAST, the ISAE-SUPAERO / ONERA aircraft sizing tool, has been implemented within OpenMDAO, the NASA open-source multidisciplinary analysis and optimization framework.
With the idea of a progressive enhancement of the multidisciplinary design analysis and a better capture of the different effects, the two pioneering elements have been studied separately.
First, the classical process has been revised to take into account the new hybrid powerplant. 
Second, a methodology has been devised to consider a radically new airframe design. Last, a design process featuring both innovative aspects has been developed to investigate a Blended Wing Body concept with distributed electric propulsion.

Regarding the design process, results show that the use of gradients in the optimization procedure speeds up the process against a gradient-free method up to 70\%. 
This is an important gain in time that facilitates designer’s tasks. For the disruptive concept performances, results have been compared to the ones obtained for a conventional A320 type aircraft based on the same top level requirements and technological horizon. 
Overall, the hybrid electric propulsion concept is interesting as it allows zero emissions for Landing/Take-Off operations, improving the environmental footprint of the aircraft: fuel can be saved for missions below a certain range. 
This limitation is associated to the presence of batteries: indeed they introduce indeed a relevant penalty in weight that cannot be countered by benefits of electrification for longer range. 
Additional simulations indicate that a Blended Wing-Body concept based on a turbo-electric only architecture is constantly performing better than the baseline within the limits of the assumptions.


{\large\textbf{Keywords:}}
Aircraft design, Blended Wing-Body, distributed electric propulsion
\\
\noindent\rule[2pt]{\textwidth}{0.5pt}

%\cleardoublepage
%\noindent\rule[2pt]{\textwidth}{0.5pt}
%{\large\textbf{Résumé ---}}
%L'augmentation du trafic aérien au cours des dernières décennies et ses prévisions constituent un défi majeur pour arriver à une croissance neutre en carbone. 
%Pour atteindre cet objectif sociétal, il est nécessaire de définir, en rupture avec les configurations  actuelles, des concepts d'avion de transport intégrant de nouvelles technologies avec au final un impact minimal sur l'environnement. 
%Ces futurs véhicules aériens reposent entre autres sur diverses interactions entre systèmes, disciplines et composants. 
%Aussi, ces travaux de recherche se focalisent sur le développement d'une méthodologie dédiée à l'exploration et à l'évaluation des performances de configurations non conventionnelles utilisant des concepts de propulsion innovants. 
%Le cas d'utilisation à considérer est l'optimisation au niveau conceptuel d'une aile volante à propulsion électrique distribuée, un concept prometteur combinant des performances aérodynamiques élevées et les avantages de la propulsion électrique.
%
%Le processus d'optimisation qui se base sur FAST, l'outil de dimensionnement avion ISAE-SUPAERO/ONERA, a été mis en œuvre dans OpenMDAO, l’environnement d’analyse et d’optimisation multidisciplinaire Open Source de la NASA. 
%Avec l'idée d'une complexité croissante de l'analyse de conception multidisciplinaire et d'une meilleure identification des différents effets, les deux éléments innovants ont été étudiés séparément. 
%Premièrement, le processus classique a été révisé pour tenir compte des systèmes de propulsion hybride. 
%Deuxièmement, une méthode a été appliquée pour estimer  le dimensionnement d’une cellule avion radicalement innovante. 
%Enfin, un processus de conception intégrant ces deux aspects inédits a été mis au point pour étudier un concept d’aile volante à propulsion électrique distribuée.
%
%En ce qui concerne le processus de conception, les résultats montrent que l’utilisation de gradients dans la procédure d’optimisation réduit les temps de calcul par rapport à une méthode sans gradient d’environ 70\%. 
%Ce gain en temps est un avantage important au niveau du processus avant-projet qui facilite les tâches du concepteur. 
%Pour les performances au niveau avion, les résultats ont été comparés à ceux obtenus pour un avion de type A320 classique, fondés sur les mêmes exigences de haut niveau et le même horizon technologique. 
%Globalement, le concept de propulsion électrique hybride est intéressant car il permet des opérations à proximité du sol (atterrissage, décollage) sans émission et d’économiser du carburant pour les missions situées en dessous d’une certaine distance franchissable. 
%Cette limitation est associée à la présence de batteries : elles introduisent en effet une pénalité de masse significative qui ne peut être annulée par les avantages de l'électrification pour de longues distances. 
%Des simulations supplémentaires indiquent qu'un concept d’aile volante fondé sur une architecture uniquement turbo-électrique consomme toujours moins de carburant que l’avion de référence dans les limites des hypothèses prises en compte.
%
%
%{\large\textbf{Mots clés :}}
%Avant-projet des avions, aile volante, propulsion electrique distribuée
%\\
%\noindent\rule[2pt]{\textwidth}{0.5pt}
%
%\vspace{0.5cm}

\end{vcenterpage}

%%% Local Variables: 
%%% mode: latex
%%% TeX-master: "../phdthesis"
%%% End: 
